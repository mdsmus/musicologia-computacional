%% modelo de documento desenvolvido pelo Grupo de Pesquisa Genos
%% www.genos.mus.br
\documentclass{article}
\usepackage[T1]{fontenc}
\usepackage[portuguese]{babel}
%\usepackage[htt]{hyphenat}
\usepackage{times}
\usepackage{color}
\usepackage{anppom2008}

\newcounter{notecounter}

\newcommand{\note}[1]{
  \addtocounter{notecounter}{1}
  \textcolor{red}{[note \arabic{notecounter}: #1]}
}

\newcommand{\rameau}{\textit{Rameau}}

\begin{document}
\graphicspath{{figs-out/}{out/}}

%%% prolegômenos

\title{Título do trabalho}
\author{Autor}{Filiação acadêmica}{e-mail}{website}

\begin{sumario}
  Resumo do texto com até 100 palavras.  
\end{sumario}

\keywords{até cinco palavras-chave que descrevam o assunto do texto}

%%% texto. Escreva a partir daqui.

\section{Sub-título 1}
\label{sec:sub-titulo-1}

Texto texto texto texto texto texto texto texto texto texto texto
texto texto texto texto texto texto texto texto texto texto texto
texto texto texto texto texto texto texto texto texto texto texto
texto texto texto texto\footnote{Nota de rodapé.}.

\subsection{Sub-título 2}
\label{sec:sub-titulo-2}

Exemplo de citação de livro \cite{turabian96:manual}, de capítulo de
livro \cite[275]{kliewer75:aspects}, de publicação em periódico
\cite{forte02:olivier}, publicação em anais de evento
\cite{kroger.ea06:processo}, e de documento eletrônico .


\section{Rameau}
\label{sec:rameau}

Todo o trabalho descrito nesse artigo foi desenvolvido usando o
sistema \rameau{} \note{citar icmc?} para análise automática de harmonia
e extração de informação musical. \rameau{}  é um framework para
implementação de algoritmos de análise harmônica de partituras
simbólicas. As partituras são guardadas em arquivos no formato
Lilypond \cite{nienhuys.ea08:lilypond}, que permite diferenciação de
notas enarmônicas e geração de partituras bem tipografadas. As notas
são extraídas das partituras e apresentadas a uma rede neural, que
associa cada conjunto de notas a um acorde.  Para esse trabalho, o
\rameau{}  foi extendido para realizar operações comuns de musicologia,
como detecção de cruzamentos, quintas ou oitavas consecutivas,
cadências frequentes, etc.

Além disso \rameau{}  possui vários outros algoritmos para análise
harmônica, como o de Pardo e Birmingham \cite{pardo.ea00:automated},
outras redes neurais e classificadores de árvores de decisão e
k-vizinhos-mais-próximos. Atualmente \rameau{}  está sendo extendido para
realizar análise funcional de qualquer peça tonal.

%%% final do texto. Não edite a partir daqui.

\renewcommand{\refname}{Referências Bibliográficas}
\bibliographystyle{kchicago}
\bibliography{writing-style,harmonic-analysis,melodic-contour,music-harmony-and-theory,programs}

%% Obs.: estas bibliotecas bibtex estão em http://git.genos.mus.br/?p=bib.git;a=summary

\end{document}
