%% modelo de documento desenvolvido pelo Grupo de Pesquisa Genos
%% www.genos.mus.br
\documentclass{article}
\usepackage{anppom2008}

\begin{document}
\graphicspath{{figs-out/}{out/}}

%%% prolegômenos

\title{Título do trabalho}
\author{Autor}{Filiação acadêmica}{e-mail}{website}

\begin{sumario}
  Resumo do texto com até 100 palavras.  
\end{sumario}

\keywords{até cinco palavras-chave que descrevam o assunto do texto}

%%% texto. Escreva a partir daqui.

\section{Sub-título 1}
\label{sec:sub-titulo-1}

Texto texto texto texto texto texto texto texto texto texto texto
texto texto texto texto texto texto texto texto texto texto texto
texto texto texto texto texto texto texto texto texto texto texto
texto texto texto texto\footnote{Nota de rodapé.}.

\subsection{Sub-título 2}
\label{sec:sub-titulo-2}

Exemplo de citação de livro \cite{turabian96:manual}, de capítulo de
livro \cite[275]{kliewer75:aspects}, de publicação em periódico
\cite{forte02:olivier}, publicação em anais de evento
\cite{kroger.ea06:processo}, e de documento eletrônico .


%%% final do texto. Não edite a partir daqui.

\renewcommand{\refname}{Referências Bibliográficas}
\bibliographystyle{kchicago}
\bibliography{writing-style,harmonic-analysis,melodic-contour,music-harmony-and-theory}

%% Obs.: estas bibliotecas bibtex estão em http://git.genos.mus.br/?p=bib.git;a=summary

\end{document}
